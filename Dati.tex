%----------------------------------------------------------------------------------------
%   USEFUL COMMANDS
%----------------------------------------------------------------------------------------

\newcommand{\dipartimento}{Dipartimento di Matematica ``Tullio Levi-Civita''}

%----------------------------------------------------------------------------------------
% 	USER DATA
%----------------------------------------------------------------------------------------

% Data di approvazione del piano da parte del tutor interno; nel formato GG/MM/AAAA
\newcommand{\dataApprovazione}{Data}

% Dati dello Studente
\newcommand{\nomeStudente}{Alberto}
\newcommand{\cognomeStudente}{Gobbo}
\newcommand{\matricolaStudente}{1170556}
\newcommand{\emailStudente}{alberto.gobbo.5@studenti.unipd.it}
\newcommand{\telStudente}{+ 39 3421741422}

% Dati del Tutor Aziendale
\newcommand{\nomeTutorAziendale}{Marco}
\newcommand{\cognomeTutorAziendale}{Bortolin}
\newcommand{\emailTutorAziendale}{m.bortolin@hunext.com}
\newcommand{\telTutorAziendale}{+ 39 0422 633882}
\newcommand{\ruoloTutorAziendale}{Software Project Manager}

% Dati dell'Azienda
\newcommand{\ragioneSocAzienda}{Hunext Software S.r.l.}
\newcommand{\indirizzoAzienda}{Via A. Volta, 23~-~31030 Casier (TV)}
\newcommand{\sitoAzienda}{www.hunext.com}
\newcommand{\emailAzienda}{software@hunext.com}
\newcommand{\partitaIVAAzienda}{P.IVA 04505810269}

% Dati del Tutor Interno (Docente)
\newcommand{\titoloTutorInterno}{Prof.}
\newcommand{\nomeTutorInterno}{Luigi}
\newcommand{\cognomeTutorInterno}{De Giovanni}

\newcommand{\prospettoSettimanale}{
     % Personalizzare indicando in lista, i vari task settimana per settimana
     % sostituire a XX il totale ore della settimana
    \begin{itemize}
        \item \textbf{Prima Settimana (XX ore)}
        \begin{itemize}
            \item Incontro con persone coinvolte nel progetto per discutere i requisiti e le richieste
            relativamente al sistema da sviluppare;
            \item Verifica credenziali e strumenti di lavoro assegnati;
            \item Presa visione dell’infrastruttura esistente;
            \item Formazione sulle tecnologie adottate;
        \end{itemize}
        \item \textbf{Seconda Settimana~-~Sottotitolo (XX ore)} 
        \begin{itemize}
            \item ;
        \end{itemize}
        \item \textbf{Terza Settimana~-~Sottotitolo (XX ore)} 
        \begin{itemize}
            \item ;
        \end{itemize}
        \item \textbf{Quarta Settimana~-~Sottotitolo (XX ore)} 
        \begin{itemize}
            \item ;
        \end{itemize}
        \item \textbf{Quinta Settimana~-~Sottotitolo (XX ore)} 
        \begin{itemize}
            \item ;
        \end{itemize}
        \item \textbf{Sesta Settimana~-~Sottotitolo (XX ore)} 
        \begin{itemize}
            \item ;
        \end{itemize}
        \item \textbf{Settima Settimana~-~Sottotitolo (XX ore)} 
        \begin{itemize}
            \item ;
        \end{itemize}
        \item \textbf{Ottava Settimana~-~Conclusione (XX ore)} 
        \begin{itemize}
            \item ;
        \end{itemize}
    \end{itemize}
}

% Indicare il totale complessivo (deve essere compreso tra le 300 e le 320 ore)
\newcommand{\totaleOre}{}

\newcommand{\obiettiviObbligatori}{
	 \item \underline{\textit{O01}}: primo obiettivo;
	 \item \underline{\textit{O02}}: secondo obiettivo;
	 \item \underline{\textit{O03}}: terzo obiettivo;
	 
}

\newcommand{\obiettiviDesiderabili}{
	 \item \underline{\textit{D01}}: primo obiettivo;
	 \item \underline{\textit{D02}}: secondo obiettivo;
}

\newcommand{\obiettiviFacoltativi}{
	 \item \underline{\textit{F01}}: primo obiettivo;
	 \item \underline{\textit{F02}}: secondo obiettivo;
	 \item \underline{\textit{F03}}: terzo obiettivo;
}