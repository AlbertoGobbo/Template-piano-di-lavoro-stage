%----------------------------------------------------------------------------------------
%   USEFUL COMMANDS
%----------------------------------------------------------------------------------------

\newcommand{\dipartimento}{Dipartimento di Matematica ``Tullio Levi-Civita''}

%----------------------------------------------------------------------------------------
% 	USER DATA
%----------------------------------------------------------------------------------------

% Data di approvazione del piano da parte del tutor interno; nel formato GG/MM/AAAA
\newcommand{\dataApprovazione}{Data}

% Dati dello Studente
\newcommand{\nomeStudente}{Alberto}
\newcommand{\cognomeStudente}{Gobbo}
\newcommand{\matricolaStudente}{1170556}
\newcommand{\emailStudente}{alberto.gobbo.5@studenti.unipd.it}
\newcommand{\telStudente}{+ 39 3421741422}

% Dati del Tutor Aziendale
\newcommand{\nomeTutorAziendale}{Marco}
\newcommand{\cognomeTutorAziendale}{Bortolin}
\newcommand{\emailTutorAziendale}{m.bortolin@hunext.com}
\newcommand{\telTutorAziendale}{+ 39 0422 633882}
\newcommand{\ruoloTutorAziendale}{Software Project Manager}

% Dati dell'Azienda
\newcommand{\ragioneSocAzienda}{Hunext Software S.r.l.}
\newcommand{\indirizzoAzienda}{Via A. Volta, 23~-~31030 Casier (TV)}
\newcommand{\sitoAzienda}{www.hunext.com}
\newcommand{\emailAzienda}{software@hunext.com}
\newcommand{\partitaIVAAzienda}{P.IVA 04505810269}

% Dati del Tutor Interno (Docente)
\newcommand{\titoloTutorInterno}{Prof.}
\newcommand{\nomeTutorInterno}{Luigi}
\newcommand{\cognomeTutorInterno}{De Giovanni}

\newcommand{\prospettoSettimanale}{
     % Personalizzare indicando in lista, i vari task settimana per settimana
     % sostituire a XX il totale ore della settimana
     Il piano di lavoro è strutturato in 3 fasi:
    \begin{itemize}
        % \item \textbf{Prima Settimana (XX ore)}
        % \begin{itemize}
        %     \item Incontro con persone coinvolte nel progetto per discutere i requisiti e le richieste
        %     relativamente al sistema da sviluppare;
        %     \item Verifica credenziali e strumenti di lavoro assegnati;
        %     \item Presa visione dell’infrastruttura esistente;
        %     \item Formazione sulle tecnologie adottate;
        % \end{itemize}
        % \item \textbf{Seconda Settimana~-~Sottotitolo (XX ore)} 
        % \begin{itemize}
        %     \item ;
        % \end{itemize}
        % \item \textbf{Terza Settimana~-~Sottotitolo (XX ore)} 
        % \begin{itemize}
        %     \item ;
        % \end{itemize}
        % \item \textbf{Quarta Settimana~-~Sottotitolo (XX ore)} 
        % \begin{itemize}
        %     \item ;
        % \end{itemize}
        % \item \textbf{Quinta Settimana~-~Sottotitolo (XX ore)} 
        % \begin{itemize}
        %     \item ;
        % \end{itemize}
        % \item \textbf{Sesta Settimana~-~Sottotitolo (XX ore)} 
        % \begin{itemize}
        %     \item ;
        % \end{itemize}
        % \item \textbf{Settima Settimana~-~Sottotitolo (XX ore)} 
        % \begin{itemize}
        %     \item ;
        % \end{itemize}
        % \item \textbf{Ottava Settimana~-~Conclusione (XX ore)} 
        % \begin{itemize}
        %     \item ;
        % \end{itemize}
        \item \textbf{Fase 1}
        \begin{itemize}
            \item Studio e formazione su C\# e Xamarin.
            \item Studio sulle soluzioni Cloud per la gestione delle Push Notification.
            \item Formazione sui tool aziendali utilizzati per la gestione dello sviluppo in team.
        \end{itemize}
        \item \textbf{Fase 2}
        \begin{itemize}
            \item Approfondimento della piattaforma Cloud da integrare nella Mobile App per la gestione delle push notification.
            \item Formazione sul backend della soluzione software collegata alla Mobile App.
            \item Creazione di un progetto di test per la sperimentazione delle Push Notification tramite Xamarin.
        \end{itemize}
        \item \textbf{Fase 3}
        \begin{itemize}
            \item Sviluppo di un POC con Xamarin per l’integrazione della Mobile App con la piattaforma Cloud di gestione delle Push Notification.
            \item Documentazione di progetto.
        \end{itemize}
    \end{itemize}
}

% Indicare il totale complessivo (deve essere compreso tra le 300 e le 320 ore)
\newcommand{\totaleOre}{320}

\newcommand{\obiettiviObbligatori}{
     \item \underline{\textit{O01}}: Studio delle possibili piattaforme Cloud per la gestione delle Push Notification.  
	 \item \underline{\textit{O02}}: Selezione della piattaforma Cloud che ai requisiti della Mobile App e dal dominio applicativo specifico.
	 \item \underline{\textit{O03}}: Sperimentazione dell'integrazione della piattaforma Cloud con la Mobile App in Xamarin.
}

\newcommand{\obiettiviDesiderabili}{
    \item \underline{\textit{D01}}: Sviluppo di un POC di gestione delle Push Notification all'interno della Mobile App.
    \item \underline{\textit{D02}}: Sviluppo di un POC per la pubblicazione delle Push Notification dal backend.
}

\newcommand{\obiettiviFacoltativi}{
    \item \underline{\textit{F01}}: Sviluppo di una strategia di gestione dei TAG per identificare l'utente destinatario della notifica.
}